\documentclass[xcolor=svgnames]{beamer}
%\documentclass[handout, xcolor=svgnames]{beamer}
\usecolortheme[named=LightSlateGrey]{structure}
\setbeamercolor{normal text}{fg=black,bg=AliceBlue}
\usetheme{Warsaw}
\useoutertheme{miniframes}

\usepackage{pgfpages}
%印出2張/1page, 直向
%\pgfpagesuselayout{2 on 1}[a4paper,border shrink=5mm]
%印出4張/1page, 橫向
%\pgfpagesuselayout{4 on 1}[a4paper, border shrink=5mm, landscape]

\usepackage{handoutWithNotes}
%\pgfpagesuselayout{4 on 1 with notes}[a4paper,border shrink=5mm]
%\pgfpagesuselayout{2 on 1 with notes landscape}[a4paper,border shrink=5mm]

\usepackage{verbatim}
\usepackage{xeCJK}
\setCJKmainfont{微軟正黑體}
\setbeamertemplate{navigation symbols}{}% 隱藏提示欄

%\setbeamercolor{normal text}{fg=Green,bg=LightGray}

%\usebackgroundtemplate{\includegraphics[width=
%\paperwidth]{test_pg.png}}

\usepackage{graphicx}
\usepackage{multimedia}
%%%%%%%%%%%%%%%%%
\begin{document}
%\logo{\includegraphics{logo.png}}
\title[Your Title\hspace{14em}\insertframenumber/\inserttotalframenumber]
{Your Title}

\author{Your Name}
\date{date}
%%%%%%%%%%%%%%%%%%%%%%%%%%%%%%%%%%%%%%%%%%%%%%%
\section{封面}
\begin{frame}
\titlepage
\end{frame}
%%%%%%%%%%%%%%%%%%%%%%%%%%%%%%%%%%%%%%%%%%%%%%%
\section{內容控制}
\begin{frame}
\frametitle{pause} %投影片標題
%%% content %%%
因為...
\pause
然後...
\pause
所以...
\end{frame}
%%%----------------------------%%%
\begin{frame}
\frametitle{item+pause} %投影片標題
\begin{itemize}
\item 第一項
\pause
\item 第二項
\pause
\item 第三項
\end{itemize}
\end{frame}
%%%----------------------------%%%
\begin{frame}
\frametitle{only} %投影片標題
\only<2->{第二張以後才會出現only}
\begin{itemize}
\item<1-> 第一項
\item<2-> 第二項
\item<3-> 第三項
\end{itemize}
\end{frame}
%%%----------------------------%%%
\begin{frame}
\frametitle{uncover} %投影片標題
\uncover<2->{第二張以後才會出現uncover}
\begin{itemize}
\item<1-> 第一項
\item<2-> 第二項
\item<3-> 第三項
\end{itemize}
\end{frame}
%%%%%%%%%%%%%%%%%%%%%%%%%%%%%%%%%%%%%%%%%%%%%%%
\section{文字變化}
\begin{frame}
\frametitle{強調文字} %投影片標題
將重點標紅字,在beamer使用\alert{\textbackslash alert}。\\ 
語法:
\textbackslash alert $\lbrace$關鍵字$\rbrace$。\\ 
在特定投影片才強調  
\alert<2>{第二張}才重要。  
\end{frame}
%%%----------------------------%%%
\begin{frame}
\frametitle{文字顏色} %投影片標題
將文字以其他顏色顯示,其語法如下:\\
$\lbrace$\textbackslash color $\lbrace$blue$\rbrace$ $\lbrace$藍色的文字$\rbrace$ $\rbrace$\\
效果如下:\\
{\color{blue}{藍色的文字}}\\[10pt]
在特定投影片才變色:\\
只有在{\color<2>{green}{第二張}}才是綠色的。\\
\begin{itemize}
\item 顏色名稱與xcolor的dvipsnames 或svgnames有關。
\end{itemize}
\end{frame}
%%%----------------------------%%%
\begin{frame}
\frametitle{文字框} %投影片標題
\begin{block}{小重點}
小重點
\end{block}

\begin{alertblock}{大重點}
大重點
\end{alertblock}
\end{frame}
%%%----------------------------%%%
\begin{frame}
\frametitle{內建定理}
\begin{theorem}
I will translate \structure{\translate[to=spanish]{theorem}} but not theorem
\end{theorem}
\end{frame}
%%%%%%%%%%%%%%%%%%%%%%%%%%%%%%%%%%%%%%%%%%%%%%%
\section{其他}
%\begin{frame}[fragile=singleslide]
\begin{frame}[fragile]
\frametitle{顯示程式碼}
\begin{block}{程式碼}
\begin{verbatim}
for i in range(0,100):
    print i
\end{verbatim}
\end{block}
\end{frame}
%%%----------------------------%%%
\begin{frame}[label=here]
\frametitle{跳到指定的投影片-目的地}
COME HERE!!!
\end{frame}
%%%----------------------------%%%
\begin{frame}
\frametitle{跳到指定的投影片-跳轉地}
\hyperlink{here}{\beamerbutton{GOOOOOOO}}
\end{frame}
%%%----------------------------%%%
\begin{frame}
\frametitle{多欄式的投影片}
\begin{columns}
\begin{column}{5cm} % 5cm高的欄
欄一
\end{column}
\begin{column}{5cm} % 5cm高的欄
欄二
\end{column}
\end{columns}
\end{frame}
%%%----------------------------%%%
\begin{frame}

\movie[width=8cm,height=4.5cm]{test}{test.avi}

\end{frame}
%%%%%%%%%%%%%%%%%%%%%%%%%%%%%%%%%%%%%%%%%%%%%%%
%%\section{}
%%\begin{frame}
%%\end{frame}
%%%%%%%%
\end{document}